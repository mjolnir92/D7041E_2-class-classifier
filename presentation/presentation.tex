\documentclass{beamer}
\usepackage{graphicx, float}
\usepackage{tikz,pgf,pgfplots,pgfplotstable,csvsimple}
%
% Choose how your presentation looks.
%
% For more themes, color themes and font themes, see:
% http://deic.uab.es/~iblanes/beamer_gallery/index_by_theme.html
%
\mode<presentation>
{
\usetheme{Madrid} % or try Darmstadt, Madrid, Warsaw, ...
\usecolortheme{seahorse} % or try albatross, beaver, crane, ...
\usefonttheme{default} % or try serif, structurebold, ...
\setbeamertemplate{navigation symbols}{}
\setbeamertemplate{caption}[numbered]
}
\usepackage[english]{babel}
\usepackage[utf8x]{inputenc}
\usepackage{verbatim, hyperref, tabularx}
%\usepackage[font=small,skip=0pt]{caption}
\title[D7041E]{2 Class Classification for password input behaviour using kNN}
\author{Mikael Hedkvist}
\institute{}
\date{\today}

\begin{document}

\begin{frame}
\titlepage
\end{frame}

\begin{frame}{Outline}
\tableofcontents
\end{frame}

\section{2 class classification problem}
\begin{frame}{2 class classfication problem}
	The data:
	\begin{itemize}
		\item Keystroke timing for typing the password \texttt{.tei5Roanl}
		\item 51 different subjects
		\item 400 inputs per subject
		\item 31 dimensions
	\end{itemize}
	The problem:
	\begin{itemize}
		\item Distinguish between subjects in a dataset of 2 subjects.
	\end{itemize}
\end{frame}

\section{kNN method}
\begin{frame}{kNN method}
	k Nearest Neighbors using L2 distance.
	\begin{enumerate}
		\item Limit data to subset containing only two subjects.
		\item Parse data into something useful:
		\begin{itemize}
			\item Labels into flat matrix of discrete values.
			\item Data into float matrix.
		\end{itemize}
		\item Two subjects = 800 rows total:	
		\begin{itemize}
			\item Training set of 600 rows.
			\item Validation set of 200 rows.
		\end{itemize}
		\item Cross fold validate to find best $k$ (3 folds of 200 rows each).
		\item Train \texttt{NearestNeighbor} with training set.
		\item Predict labels of validation set using L2 distance.
	\end{enumerate}
\end{frame}

\section{Results}
\begin{frame}{Results}
	\begin{figure}[h]
        \centering
        \begin{tikzpicture}[scale = 1.0]
        \begin{axis}[
            ylabel = accuracy,
            xlabel = k,
            legend pos = north east]
        \addplot table[y=acc, x=k, col sep = comma]{csv/k-acc_linear_s002-s003.csv};
        \addlegendentry{s002 \& s003}
        \end{axis}
        \end{tikzpicture}
        \caption{Accuracy for subjects s002 and s003 for $k=\{1-200\}$.}
        \label{fig:1}
    \end{figure}
\end{frame}

\begin{frame}{Results}
	\begin{figure}[h]
        \centering
        \begin{tikzpicture}[scale = 1.0]
        \begin{axis}[
            ylabel = accuracy,
            xlabel = k,
            legend pos = north east]
        \addplot table[y=acc, x=k, col sep = comma]{csv/k-acc_linear_s003-s004.csv};
        \addlegendentry{s003 \& s004}
        \end{axis}
        \end{tikzpicture}
        \caption{Accuracy for subjects s003 and s004 for $k=\{1-200\}$.}
        \label{fig:1}
    \end{figure}
\end{frame}

\begin{frame}{Results}
	\begin{figure}[h]
        \centering
        \begin{tikzpicture}[scale = 1.0]
        \begin{axis}[
            ylabel = accuracy,
            xlabel = k,
            legend pos = north east]
        \addplot table[y=acc, x=k, col sep = comma]{csv/k-acc_linear_s004-s005.csv};
        \addlegendentry{s004 \& s005}
        \end{axis}
        \end{tikzpicture}
        \caption{Accuracy for subjects s004 and s005 for $k=\{1-200\}$.}
        \label{fig:1}
    \end{figure}
\end{frame}

\section{Conclusions}
\begin{frame}{Conclusions}
	According to the results low values for $k$ generally produces the best results, and even for high values the accuracy converges only a few percent below the maximum.

	kNN proved to have generally high accuracy for a 2 class classification problem, however...
	\begin{itemize}
		\item accuracy is dependant on the subjects:
		\begin{itemize}
			\item more distinct patterns are easier to predict.
			\item subjects with similar behaviour will yield lower accuracy.
		\end{itemize}
		\item this method can't detect an unknown attacker:	
		\begin{itemize}
			\item new input will always be labeled to the closest match in the known data.
			\item detecting if the new subject is not in the known data is a different problem.
			\item making clusters for subjects and setting max deviation thresholds would be more practical for this.
		\end{itemize}
	\end{itemize}
\end{frame}

\section{The code}
\begin{frame}{The code}
	\texttt{https://github.com/mjolnir92/D7041E\_2-class-classifier.git}
\end{frame}

\end{document}
